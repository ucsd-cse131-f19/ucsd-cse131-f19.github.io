\documentclass[10pt, oneside]{article}
\usepackage[letterpaper, scale=0.9, centering]{geometry}
\usepackage{graphicx}			
\usepackage{url,hyperref}
\usepackage{amssymb,amsmath,mathpartir}
\usepackage{currfile,xstring,multicol,array}
\usepackage[dvipsnames]{xcolor}
\usepackage[labelformat=empty]{caption}

\title{UCSD CSE131 F19 -- Substitution, Variables, Single-Stepping, and Functions}

\usepackage{listings}

\lstset{
  basicstyle=\ttfamily,
  mathescape,
  columns=fullflexible
}

\pagestyle{empty}
\setlength{\parindent}{0em}
\setlength{\parskip}{1em}
\begin{document}
\maketitle 

\subsection*{Describing Evaluation}

One of the things we often need to do is describe how a programming language
feature ought to work without writing out precisely how it compiles. This is
important, because we may want to implement the language in multiple ways, or
for multiple processor types, and need to specify what the language should do
independent of a particular platform.

We've worked with a few different strategies for doing this, but two major
ones. First, we've appealed to step-by-step algebraic substituion by example,
and second, we've written out descriptions in precise English. For example,
in Copperhead, we describe \texttt{set} expressions with:

\begin{quote}
  \texttt{set} expressions update the value of a variable. The behavior
  of a \texttt{set} expression is to evaluate its subexpression, then set the
  value of the named variable to the result of that subexpression. The result
  of the entire \texttt{set} expression is the new value, and its effect is to
  make future accesses of that variable get the updated value.
\end{quote}

There's a lot going on there, and some of it is subtle. In fact, it's not
clear that this description and the algebraic substitution idea are
compatible! Let's look at a simple example:

\begin{lstlisting}
(let (x 5) (set x 10) x)
\end{lstlisting}

In our algebraic model, we might have said this takes a step by replacing
occurrences of x with 5:

\begin{lstlisting}
(set 5 10) 5
\end{lstlisting}

This is nonsense, for two reasons. First, it doesn't make any sense to set
the value of 5 to the value of 10 -- what does that mean? Second, this
program ought to evaluate to 10, but if we just evaluate the sequence of
expressions in order, we'd get 5. The step-by-step rule seems to have been
incorrect in substituting the value of x when it could be updated by a set
before being used. In fact, this substitution idea seems to be at odds with
the English description which refers to ``future accesses of that variable''.
If we substitute away all the variable accesses as soon as we define the
variable, we can't talk about accessing it at different points in time.

It's possible to reconcile these two views, and there are a few options. One
that we're going to use involves leaving some context around during the
step-by-step evaluation so we can accurately process updates to variables.
The idea we'll use is to keep the \texttt{let} bindings surrounding its body
as they evaluate step-by-step, and refer back to the set of bindings for
access and update. So we could think of the above program as taking these
steps:

\begin{lstlisting}
    (let (x 5) (set x 10) x)
$\textrm{We focus on the set expression while leaving the let in place}$
$\rightarrow$ (let (x 5) $\fbox{\texttt{(set x 10)}}$ x)}
$\textrm{The set expression changes the immediately-surrounding let's x binding}$
$\rightarrow$ (let (x $\fbox{\texttt{10}}$}) 10 x)}
$\textrm{The variable expression looks up the x binding in the immediately-surrounding let}$
$\rightarrow$ (let (x $\fbox{\texttt{10}}$}) 10 $\fbox{\texttt{10}}$)}
$\textrm{The body evaluates to the last expression's value}$
$\rightarrow$ (let (x 10}) 10)
$\textrm{When a let expression has just one value in its body, that's the result of the whole expression, and we're done with this x}$
$\rightarrow$ 10
\end{lstlisting}

This is an interpretation that uses some context---the let expression's
bindings---to track information about variables' values during evaluation.
This is a way to write down in text what you may have written down in other
contexts as a stack frame with a box mapping variables to their values. This
idea composes well with some related ideas for loops. For example, consider
this program, that sums the numbers from 1 to 9:

\begin{lstlisting}
    (let (x 1)
      (let (sum 0)
        (while (< x 10)
          (set sum (+ sum x))
          (set x (+ x 1)))
        sum))
$\textrm{The idea here is that a while loop steps to an if that either runs one iteration and then loops, or evaluates to false}$
$\rightarrow$ (let (x 1)
      (let (sum 0)
        (if (< x 10)
            (
              (set sum (+ sum x))
              (set x (+ x 1))
              (while (< x 10)
                (set sum (+ sum x))
                (set x (+ x 1))
              )
            )
            false)
        sum))
$\textrm{Look up current value of x and compare}$
$\rightarrow$ (let (x 1)
      (let (sum 0)
        (if $\fbox{\texttt{true}}$
            (
              (set sum (+ sum x))
              (set x (+ x 1))
              (while (< x 10)
                (set sum (+ sum x))
                (set x (+ x 1))
              )
            )
            false)
        sum))
$\textrm{Reduce if to its then case}$
$\rightarrow$ (let (x 1)
      (let (sum 0)
        (set sum (+ sum x))
        (set x (+ x 1))
        (while (< x 10)
          (set sum (+ sum x))
          (set x (+ x 1))
        )
        sum))
$\textrm{Look up sum and x}$
$\rightarrow$ (let (x 1)
      (let (sum 0)
        (set sum (+ $\fbox{\texttt{0}}$ $\fbox{\texttt{1}}$))
        (set x (+ x 1))
        (while (< x 10)
          (set sum (+ sum x))
          (set x (+ x 1))
        )
        sum))
$\textrm{Update the sum variable}$
$\rightarrow$ (let (x 1)
      (let (sum $\fbox{\texttt{1}}$)
        $\fbox{\texttt{1}}$
        (set x (+ x 1))
        (while (< x 10)
          (set sum (+ sum x))
          (set x (+ x 1))
        )
        sum))
$\textrm{Look up then update x}$
$\rightarrow$ (let (x $\fbox{\texttt{2}}$)
      (let (sum 1)
        1
        $\fbox{\texttt{2}}$
        (while (< x 10)
          (set sum (+ sum x))
          (set x (+ x 1))
        )
        sum))
$\textrm{Process the while loop again}$
$\rightarrow$ (let (x 2)
      (let (sum 1)
        1
        2
        (if (< x 10)
            (
              (set sum (+ sum x))
              (set x (+ x 1))
              (while (< x 10)
                (set sum (+ sum x))
                (set x (+ x 1))
              )
            )
            false)
        sum))
$\textrm{and so on. The final state will look something like this:}$
$\rightarrow$ (let (x 10)
      (let (sum 45)
        1  2  3 3  6 4
        10 5 15 6 21 7
        28 8 36 9 45 10
        false
        45))
\end{lstlisting}

It's helpful to be aware that we can write programs out with variable update
in this way, because it means we don't have to throw out all the work we did
understanding substitution. We simply need to enhance our model to
accommodate using let as a binding context that stores the current value of a
variable.

Of course, this is still just one way to describe evaluation. The English
descriptions are still valuable, and sketching examples with pictures of
stack frames is also valuable, and talking about the specific generated
instructions and memory layout is also valuable! We will continue to use
whichever one is appropriate and helpful for understanding a particular
topic, with the knowledge that we can shift between them as necessary.

This also harmonizes our understanding of programs that rely on loops and
variable update with the algebraic stepping approach. This is useful because
it gives us some chances to see a more unified framework that explains both
the function-and-recursion focused style of OCaml and Haskell as well as the
loop-and-variable style of C, Python, and Java.

\subsection*{Describing Function Evaluation}

Many languages have functions, and the reasons for having them are several
(we could list more beyond these):

\begin{enumerate}
\item Encapsulating a computation into a limited scope and span of code
\item Avoiding code repetition
\item Implementing recursive algorithms
\item Describing an operation that should happen at a future time (when the
function is applied), potentially multiple times
\end{enumerate}

The essence of functions is twofold:

\begin{enumerate}
\item The separation of a definition of an expression from its use
\item The parameterization of an expression with names whose values are
chosen when the function is applied
\end{enumerate}

This motivates a design for functions that has two pieces of syntax -- one
for defining functions and one for using them.\footnote{We omit types in this
description for simplicity, since the types by design don't affect how the
program \emph{runs}!}

\[
\begin{array}{ll}
\begin{array}{lrl}
e & := & \ldots \textrm{other expressions} \ldots \mid \texttt{($f\ e$)} \\
d & := & \texttt{(def ($f$ $x$) $e$)} \\
\end{array}
\end{array}
\]

Or, in Ocaml:

\begin{lstlisting}
type expr =
  | EApp of string * expr

type def =
  | DFun of string * string * expr

type prog = def list * expr
\end{lstlisting}

Note that these functions have only a \textit{single} argument. The
descriptions here do generalize to multi-argument functions with some work.

So a function definition for the absolute value function is:

\begin{lstlisting}
(def (abs x)
  (if (< x 0) (* -1 x) x))
\end{lstlisting}

and an application of it is:

\begin{lstlisting}
(abs 9)
\end{lstlisting}

It's relatively straightforward to agree that it should evaluate to
\texttt{9}, and that \texttt{(abs -3)} should evaluate to \texttt{3}. We
could describe this in a few ways. One is to use an English sentence like:

\begin{quote} A function application evaluates by first evaluating its
argument, then finding the definition that matches the given function name,
and evaluating the definition's body with the parameter name set to the value
of the argument. The result of the body's evaluation is the result of the
entire function application.\end{quote}

It's also interesting to think about how to define it in terms of algebraic
stepping. There are a few ways we could do this, but one that uses some of
what we discussed in the last section is helpful here. A function application
works by stepping to a \texttt{let} expression:


\begin{lstlisting}
    (abs 9)
$\rightarrow$ (let (x 9) (if (< x 0) (* -1 x) x))
$\rightarrow$ (let (x 9) (if false (* -1 x) x))
$\rightarrow$ (let (x 9) x)
$\rightarrow$ 9
\end{lstlisting}

That is, when we have a function application like \texttt{($f\ e_a$)}, and a
definition with a matching name \texttt{(def $f$ ($x$:$\tau_1$):$\tau_2$ $e_b$)}, it
evaluates by taking a step to \texttt{(let ($x$ $e_a$) $e_b$)}.

This is a helpful description because it naturally generalizes to functions
that call other functions, including recursive ones, and including functions
that might use \texttt{set} on their arguments.

\begin{lstlisting}
(def (sum x)
  (if (< x 1) 0 (+ x (sum (- x 1)))))
\end{lstlisting}

\begin{lstlisting}
    (sum 3)
$\rightarrow$ (let (x 3) (if (< x 1) 0 (+ x (sum (- x 1))))))
$\rightarrow$ (let (x 3) (if false 0 (+ x (sum (- x 1))))))
$\rightarrow$ (let (x 3) (+ x (sum (- x 1)))))
$\rightarrow$ (let (x 3) (+ 3 (sum 2))))
$\rightarrow$ (let (x 3) (+ 3 (let (x 2) (if (< x 1) 0 (+ x (sum (- x 1))))))))
$\rightarrow$ (let (x 3) (+ 3 (let (x 2) (if false 0 (+ x (sum (- x 1))))))))
$\rightarrow$ (let (x 3) (+ 3 (let (x 2) (+ x (sum (- x 1))))))
$\rightarrow$ (let (x 3) (+ 3 (let (x 2) (+ 2 (let (x 1) (if (< x 1) 0 (+ x (sum (- x 1)))))))))
$\rightarrow \ldots$
$\rightarrow$ (let (x 3) (+ 3 (let (x 2) (+ 2 (let (x 1) (+ 1 (let (x 0) 0)))))))
$\rightarrow \ldots$
$\rightarrow$ 6
\end{lstlisting}

Note that in this example, it's crucial that variable lookup finds the
instance of x that is the closest to the use -- there is space for one x
created per function call with the \texttt{let}. This matches the behavior of
\texttt{let} that we've used since the beginning of the course.

This description also accounts for a different version of sum that uses a loop:

\begin{lstlisting}
(def (sum x)
  (let (sum 0)
    (while (> x 0)
      (set sum (+ sum x))
      (set x (- x 1)))
    sum))
\end{lstlisting}

\begin{lstlisting}
(sum 3)
$\rightarrow$
(let (x 3)
  (let (sum 0)
    (while (> x 0)
      (set sum (+ sum x))
      (set x (- x 1)))
    sum))
$\rightarrow \ldots$
\end{lstlisting}

This continues with a similar pattern to the earlier sum example.

\subsection*{Compiling Function Evaluation}

Of course, our eventual goal in this course is to turn source programs into
executables by way of x86-64 assembly. The above discussion will help us in
several ways.

First, we assume a few extra steps as setup -- first we parse all the
definitions into a list of \texttt{def} along with a final expression as an
\texttt{expr}. We compile the final expression which is the entry point for
the program, and pass the definitions list to \texttt{e\_to\_is} so we can
use definition information during compilation.

\begin{lstlisting}
type prog = def list * expr
let parse_def (sexp : Sexp.t) : def = ... parse ...
let parse_program (sexps : Sexp.t list) : prog = ... parse ...

let rec e_to_is (e : expr) (si : int) (env : tenv) (defs : def list) : string list =
... other cases as before ...
| EApp(f, arg) ->

let compile (program : Sexp.t list) : string =
  let (defs, body) = parse_program program in
  let instrs = e_to_is body 1 [] defs in
\end{lstlisting}

\subsubsection*{A First Try}

The step-by-step rule we gave above suggests a really simple first cut at
compiling functions -- just use let! So we could try something like this:

\begin{lstlisting}
let rec e_to_is (e : expr) (si : int) (env : tenv) (defs : def list) : string list =
... other cases as before ...
| EApp(f, arg) ->
  match find_def f def_env with
    | None -> failwith "No such function definition"
    | Some(DFun(name, arg_name, arg_typ, ret_typ, body)) ->
      compile_expr (ELet(arg_name, arg, body))
\end{lstlisting}


We can try it on a simple program as input:

\begin{lstlisting}
(def (f x)
  (+ x 2))
(f 10)
\end{lstlisting}

If we compile and run this, we get the right answer!

Other examples work, as well, including calling the function twice, as in
\texttt{(+ (f 10) (f 3))}, or using \texttt{(f (f 7))} as the main
expression.

Even more complex examples work, like calling a function from within another
function:

\begin{lstlisting}
(def (g y)
  (+ 5 y))
(def (f x)
  (+ x (g x)))
(f 42)
\end{lstlisting}



Awesome! This will extend well to functions that contain \texttt{while} loops
and variable assignment. Clearly, we are done and can ship the compiler to
our users, confident that they will now be able to write and use functions!

\vspace{10em}

Wait a minute. This is weird.

We ``implemented'' functions, but somehow managed to avoid ever using the
\texttt{call} and \texttt{ret} instructions that seem purposefully built for
function call and return.

We ``implemented'' functions, but never talked about the stack pointer or
where the code for a function goes.

This violates a lot of intuition we probably have from looking at compiled C
programs.

Let's try just one more function:

\begin{lstlisting}
(def (sum n)
  (if (< n 1) 0 (+ n (sum (+ n -1)))))
(sum 3)
\end{lstlisting}

\begin{verbatim}
> make sum.run
./compile sum.int > sum.s
Fatal error: exception Stack overflow
make: *** [sum.s] Error 2
rm sum.s
\end{verbatim}

That's a stack overflow {\it in the compiler} -- the assembly didn't even get
generated. When we compile the \texttt{EApp("sum", ENum(3))} expression, we
look up the definition of {\tt sum}, which has a use of {\tt sum} in its
body. Then we recursively call {\tt e\_to\_is} with a new let expression
containing that call. When that use of {\tt sum} is visited by the compiler,
we generate a let with the body of {\tt sum} again, and the cycle repeats.
Eventually we reach Ocaml's maximum stack depth and the compiler crashes.

Since in general we can't know how many recursive calls will happen,
especially in a case like {\tt sum} where there is an {\tt if} separating the
base case and the recursive case, we need to do something more sophisticated.
It's worth pointing out that for functions that aren't recursive (or never
call a recursive function) the strategy of copying the expression actually
produces working programs! It's an optimization strategy called {\bf
inlining} that trades larger generated code for decreasing the number of
function calls that happen.

However, for recursive calls, we have to do something different, and this
motivates compiling definitions separate from calls, and the idea of a {\bf
calling convention}. We still want to get the kind of behavior we saw with
let above, but have to approach the implementation differently.

A simple such {\bf calling convention} is implemented in
\url{https://ucsd-cse131-f19.github.io/lectures/10-22-lec8/compile.ml} and
demonstrated in
\url{https://ucsd-cse131-f19.github.io/lectures/10-22-lec8/calling.pdf}.

\end{document}