
\documentclass[10pt, oneside]{article}
\usepackage[letterpaper, scale=0.9, centering]{geometry}
\usepackage{graphicx}			
\usepackage{url,hyperref}
\usepackage{amssymb,amsmath,mathpartir}
\usepackage{currfile,xstring,multicol,array}
\usepackage[dvipsnames]{xcolor}
\usepackage[labelformat=empty]{caption}

\title{UCSD CSE131 F19 -- Garter}

\usepackage{listings}

\lstset{
  basicstyle=\ttfamily,
  mathescape,
  columns=fullflexible
}

\setlength{\parindent}{0em}
\setlength{\parskip}{1em}
\begin{document}
\maketitle 

{\bf Checkpoint Due Date:} 11pm Wednesday, November 27 \hspace{2em} {\bf Final Due Date:} 11pm {\bf Thursday} December 5

The specific features listed for the checkpoint are {\bf Open to
Collaboration} (detailed below), and the rest is {\bf Closed to
Collaboration}.

You will implement memory management atop a type-checked language with
heap-allocated data and functions.

Classroom: FILL \hspace{1em} Github: FILL

\section*{Syntax}

The concrete syntax and type language for Garter is below. We use $\cdots$ to
indicate \textit{zero or more} of the previous element. There are
\fbox{boxes} around the new pieces of concrete syntax.

\[
\begin{array}{ll}
\begin{array}{lrl}
e & := & n \mid \texttt{true} \mid \texttt{false} \mid x \\
  & \mid  & \texttt{(let (($x\ e$) ($x\ e$) $\cdots$) $e\ e \cdots$)} \\
  & \mid  & \texttt{(if $e\ e \ e$)} \\
  & \mid  & \texttt{($op_2$ $e\ e$)} \mid \texttt{($op_1$ $e$)} \\
  & \mid  & \texttt{(while $e\ e\ e \cdots$)} \mid \texttt{(set $x\ e$)} \\
  & \mid  & \texttt{($f\ e\cdots$)} \mid \fbox{\texttt{(null $\tau$)}} \\
  & \mid  & \fbox{\texttt{(get $e$ $n$)}} \mid \fbox{\texttt{(update $e$ $n$ $e$)}} \\
d & := & \texttt{(def $f$ ($x\ $:$\ \tau \cdots$) :$\ \tau$ $e\ e \cdots$)} \\
  & \mid  & \fbox{\texttt{(data $C$ ($\tau \cdots$))}} \\
p & := & d \cdots e \\
op_1 & := & \texttt{add1} \mid \texttt{sub1} \mid \texttt{isNum} \mid \texttt{isBool} \mid \texttt{print} \\
op_2 & := & \texttt{+} \mid \texttt{-} \mid \texttt{*} \mid \texttt{<} \mid \texttt{>} \mid \texttt{==} \mid \fbox{\texttt{=}} \\
n & := & \textrm{63-bit signed number literals} \\
x,f,C & := & \textrm{variable, function, and constructor names} \\
\end{array}
&
\begin{array}{lrl}
\tau & := & \texttt{Num} \mid \texttt{Bool} \mid C \\
\delta & := & \texttt{fun} \mid \texttt{data} \\
\Delta & := & \{\delta\ f: \tau \cdots \rightarrow \tau, \cdots\} \\
\Delta[f] & \textit{means} & \textit{look up the type of } f \textit{ in } \Delta \\
\Gamma & := & \{x:\tau,\cdots\} \\
\Gamma[x] & \textit{means} & \textit{look up the type of } x \textit{ in } \Gamma \\
(x,\tau)::\Gamma & \textit{means} & \textit{add } x \textit{ to } \Gamma \textit{ with type } \tau \\
\Delta;\Gamma \vdash e : \tau & \textit{means} & \textit{with definitions } \Delta \textit{ and env } \Gamma, e \textit{ has type } \tau \\
\Delta \vdash_d d : \checkmark & \textit{means} & \textit{with definitions } \Delta \textit{ the definition } d \textit{ type-checks} \\
\vdash_p p : \checkmark & \textit{means} & \textit{the program } p \textit{ type checks } \\
\end{array}
\end{array}
\]


\section*{Semantics}

The semantics here are all provided for you, we describe them so you'll be
able to write accurate tests.

\subsection*{Data Definitions, Construction, and Manipulation}

The main new feature in Garter is data definitions \texttt{(data $C$
($\tau \cdots$))}, where $s$ is the name of the data definition and the types
$\tau \cdots$ are the types of the elements stored in instances of the data
definition. Elements are accessed and updated positionally with \emph{fixed}
(not computed, as with arrays) numeric indices using \texttt{(get $e$ $n$)}
and \texttt{(update $e$ $n$ $e$)}.\footnote{As an analogy, data definitions
are somewhat like structs in C, but use positional lookup instead of names;
as another analogy, data definitions are like tuples in OCaml and we can
match on them by statically known positions using functions like {\tt fst}
and {\tt snd}, but not compute the position of lookup.} The syntax for
function applications is used to construct new data instances.

As an example, this program evaluates to 67:

\begin{lstlisting}
(data Pair (Num Num))
(let (
    (p1 (Pair 4 5))
    (p2 (Pair 4 5))
    (p3 p1)
    )
  (update p1 0 11)
  (update p2 1 56)
  (+ (get p3 0) (get p2 1)))
\end{lstlisting}

\subsection*{Printing Data Instances}

In Egg-Eater, locations (referring to instances of data) are a new kind of
value that can be printed, just like numbers and booleans.

When an instance is printed, it should print in the format

\begin{lstlisting}
($C$ $v_1$ $v_2$ $\cdots$)
\end{lstlisting}

Where $C$ is the name of the constructor used to create it, and values $v_1$
and $v_2$ are the printed form of the values stored in its fields, separated
by spaces.

For example:

\begin{lstlisting}
(data Pair (Num Num))
(data PairOfPairs (Pair Pair))
(let ((p (PairOfPairs (Pair (+ 1 2) 6) (Pair (add1 6) 8))))
  p)

# prints:
(PairOfPairs (Pair 3 6) (Pair 7 8))
\end{lstlisting}

\subsection*{Equality}

There two types of equality in Egg-Eater, reflecting the new nuances of
heap-allocated data. The first, \texttt{==}, behaves as before on existing
values, and on locations referring to instances of data, returns
\texttt{true} if the \emph{locations} are identical. The second, \texttt{=},
behaves as before on existing values, and on locations returns \texttt{true}
if the two instances came from the same constructor and the \emph{contents}
of those locations are all equal according to \texttt{=}.

For example:

\begin{lstlisting}
(data Pair (Num Num))
(data PairOfPair (Pair Pair))
(data Point (Num Num))
(let (
  (p1 (Pair 3 4))
  (p2 (Pair 3 4))
  (p3 (Point 3 4))
  (p4 (Point 3 5))
  (pp12 (PairOfPair p1 p2))
  (pp21 (PairOfPair p2 p1))
  )
  (print (=  p1 p2)) ; true, same constructor and contents
  (print (== p1 p2)) ; false, different locations
  (print (=  p1 p3)) ; false, different constructors
  (print (== p1 p3)) ; false, different locations
  (print (=  p3 p4)) ; false, different contents
  (print (=  pp12 pp21)) ; true, same (nested) contents
  0
  )
\end{lstlisting}

\section*{Type Checking}

Egg-Eater has essentially the same type rules as Diamondback for expressions.
The definitions environment $\Delta$ is constructed with the types of the
constructors for data definitions as well as function definitions; these are
distinguished by either \texttt{data} or \texttt{fun} before the
name.\footnote{As an implementation note, we found it useful to simply pass
around the entire list of definitions in several functions.} As an example,
the definition \texttt{(data Point (Num Num))} would appear in $\Delta$ as
$\texttt{data Point} : (\texttt{Num } \texttt{Num} \rightarrow \texttt{Point})$.
There is a new rule for each new syntactic form, except for \texttt{data}
definitions which don't need separate type checking.

\begin{mathpar}
\inferrule*[Left=TR-Null]
{\Delta[\texttt{data}\ $C$] = (\tau_1 \cdots \rightarrow \tau_r) }
{\Delta;\Gamma \vdash \texttt{(null $C$)} : C}
\\
\inferrule*[Left=TR-Get]
{\Delta;\Gamma \vdash e : $\ C$ \and \Delta[\texttt{data}\ $C$] = (\tau_1 \cdots \tau_n \tau_{n+1} \cdots \rightarrow \tau_r) }
{\Delta;\Gamma \vdash \texttt{(get $e$ $n$)} : \tau_n}
\\
\inferrule*[Left=TR-Update]
{\Delta;\Gamma \vdash e : $\ C$ \and \Delta[\texttt{data}\ $C$] = (\tau_1 \cdots \tau_n \tau_{n+1} \cdots \rightarrow \tau_r) \and
\Delta;\Gamma \vdash e_v : \tau_n}
{\Delta;\Gamma \vdash \texttt{(update $e$ $n$ $e_v$)} : \tau_n}
\end{mathpar}

There are a few important features here.

\begin{itemize}
  \item The \texttt{null} expression comes with a type that it should be
  treated as. The type checker simply checks that this annotation is some
  \texttt{data} type and treats the \texttt{null} value as that type. This
  allows us to construct instances of recursively-defined datatypes like
  \texttt{(Link (Num Link))}.
  \item In TR-Get and TR-Update, we check that the first expression has a
  type of some data definition $C$. The types before the $\rightarrow$ are
  the types of the fields or elements listed in the data definition.
  \item We assume the existing rule for TR-App in applications, which simply
  checks that values with the right types are present in order according to
  the data definition (just like for function calls).
\end{itemize}

\section*{Application Binary Interface}

\subsection*{Value and Heap Layout}

The value layout is extended to keep track of information needed in garbage
collection:

\begin{itemize}
\item {\tt 0xXXXXXXXXXXXXXXX[xxx1]} - Number
\item {\tt 0x000000000000000[0110]} - True
\item {\tt 0x000000000000000[0010]} - False
\item {\tt 0x000000000000000[0000]} - Null
\item {\tt 0xXXXXXXXXXXXXXXX[x000]} - Data Reference, an address of a data instance on the heap
laid out as follows (each set of {\tt[]} is one 8-byte word)

  {\tt [ GC word ][ name reference ][ element count n ][ value 1 ][ value 2 ] ... [ value n ]}
\end{itemize}

The use of the GC word is completely up to your memory management
implementation and is always initialized to 0 (see below). The name reference
is the address of a C string that holds the struct's name (essentially a {\tt
char*}) used in printing and equality. The element count tracks the number of
elements stored in the data value.

As an example, consider this program:

\begin{lstlisting}
(data Pair (Num Num))
(let (
    (p1 (Pair 4 5))
    (p2 (Pair 6 7))
    (p3 p1)
    )
    ...)
\end{lstlisting}

The stack word for {\tt p1} would hold a value like {\tt 0x00000000ABCDE120}, where
at address {\tt 0x00000000ABCDE120} would be stored:

\begin{verbatim}
0x00000000ABCDE230 : [ 0x0000000000000000 ] ; gc word
                     [ 0x00000000NAMEADDR ] ; address of "Pair"
                     [ 0x0000000000000002 ] ; count of elements
                     [ 0x0000000000000009 ] ; representation of 4
                     [ 0x000000000000000B ] ; representation of 5
\end{verbatim}

Where at {\tt 0xNAMEADDR} we would find the characters {\tt Pair\verb+\+0},
and 9 and 11 are the representations of 4 and 5. At the stack word for {\tt
p3} we would also find {\tt 0x00000000ABCDE120}. At the stack word for {\tt
p2} we should expect to find a different address, say {\tt
0x00000000ABCDE230}, with a similar layout but different values:

\begin{verbatim}
0x00000000ABCDE230 : [ 0x0000000000000000 ] ; gc word
                     [ 0x00000000NAMEADDR ] ; address of "Pair"
                     [ 0x0000000000000002 ] ; count of elements
                     [ 0x000000000000000D ] ; representation of 6
                     [ 0x000000000000000F ] ; representation of 7
\end{verbatim}

\subsection*{Calling Convention}

We use a calling convention similar to the one discussed in class, so at any
given moment there are a number of function calls on the stack, each with
arguments and local variables.

\begin{figure}
\begin{verbatim}
        [   UNUSED SPACE    ] <- stack_top
        ---------------------
        [local var N        ]
        [...                ]
        [local var 1        ]   these locals and args are
        [arg 1              ]   for the topmost active
        [...                ]   function call
        [arg N              ]
        [prev rsp value     ]
rsp ->  [return address     ] <- first_frame
        ---------------------
                 ...            
        ---------------------
        [local var N        ]   these locals and args are
        [...                ]   for a current active
        [local var 1        ]   function call
        [arg 1              ]
        [...                ]
        [arg N              ]
        [prev rsp value     ]
        [return address     ]
        ---------------------
        [local var N        ]  these locals are for
        [...                ]  the main expression
        [local var 1        ]
        [ret ptr to main    ] <- STACK_BOTTOM
\end{verbatim}
\end{figure}

Some important highlights:

\begin{itemize}
\item On the right, we show the addresses stored in the arguments given to
{\tt try\_gc} which are passed on to the {\tt gc} function you will write.
This includes {\tt stack\_top}, which is equal to {\tt rsp - (stackloc si)},
{\tt first\_frame}, which is equal to {\tt rsp}, and {\tt STACK\_BOTTOM},
which is a global that refers to the original value of {\tt rsp} right after
calling {\tt our\_code\_starts\_here}. We will say more about each of these
in the next section.
\item We made sure the compiler implements the invariant that {\tt rsp -
(stackloc si)} will always refer to the word above the topmost valid value,
and that there won't be any invalid values in the local variables or the
arguments on the stack.
\end{itemize}


\end{document}